% Copyright © 2014 Martin Ueding <dev@martin-ueding.de>

\chapter{Herleitung 1D-Wellenfunktion}

\newcommand\br[1]{\parencite[(#1)]{Busch/Two_Cold}}

Analog zu \br2:
\[
    H_\text{rel} =
    \underbrace{- \frac12 \nabla_x^2 + \frac 12 x^2}_{H_\text{osz}} - \frac2{a_0} \delta(x)
\]

Analog \br3:
\[
    \del{H_\text{osz} - \frac2{a_0} \delta(x)} \Psi(x) = E \Psi(x)
\]

Benutze ebenfalls \br4:
\[
    \Psi(x) = \sum_{n=0}^\infty c_n \phi_n(x)
\]

Hier gibt es keinen Drehimpuls $l$. Aber $\phi_n$ mit ungeradem $n$ tragen
nicht bei, da diese bei $x = 0$ einen Nulldurchgang haben.

Einsetzen, analog zu \br5:
\[
    \del{H_\text{osz} - E - \frac2{a_0} \delta(x)}
    \sum_{n=0}^\infty c_n \phi_n(x)
    = 0
\]


% vim: ft=tex spell spelllang=de tw=79
