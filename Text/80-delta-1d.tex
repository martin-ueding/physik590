% Copyright © 2014 Martin Ueding <dev@martin-ueding.de>

\chapter{Herleitung 1D-Wellenfunktion}

\newcommand\br[1]{\parencite[(#1)]{Busch/Two_Cold}}

Analog zu \br2:
\[
    H_\text{rel} =
    \underbrace{- \frac12 \nabla_x^2 + \frac 12 x^2}_{H_\text{osz}} - \frac2{a_0} \delta(x)
\]

Analog \br3:
\[
    \del{H_\text{osz} - \frac2{a_0} \delta(x)} \Psi(x) = E \Psi(x)
\]

Benutze ebenfalls \br4:
\[
    \Psi(x) = \sum_{n=0}^\infty c_n \phi_n(x)
\]

Hier gibt es keinen Drehimpuls $l$. Aber $\phi_n$ mit ungeradem $n$ tragen
nicht bei, da diese bei $x = 0$ einen Nulldurchgang haben.

Einsetzen, analog zu \br5:
\[
    \del{H_\text{osz} - E - \frac2{a_0} \delta(x)}
    \sum_{n=0}^\infty c_n \phi_n(x)
    = 0
\]

\[
    \sum_{n=0}^\infty (E_n - E) c_n \phi_n(x)
    \frac2{a_0} \delta(x)
    \sum_{n=0}^\infty c_n \phi_n(x)
    = 0
\]

Dies wird nun auf $\phi_m^*$ projiziert, i.\,e. Multiplikation von rechts und
Integration über $x$:
\[
    \int_{-\infty}^\infty \dif x \, \sum_{n=0}^\infty
    (E_n - E) c_n \phi_m^*(x) \phi_n(x)
    - 
    \int_{-\infty}^\infty \dif x \,  \frac2{a_0} \phi_m^*(x) \delta(x)
    \sum_{n=0}^\infty c_n \phi_n(x) = 0
\]

Mit
\[
    \int_{-\infty}^\infty \dif x \phi_m^*(x) \phi_n(x) = \braket{m|n} =
    \delta_{mn}
\]
folgt
\[
    (E_m - E) c_m - \frac2{a_0} \phi_m^*(0) \sum_{n=0}^\infty c_n \phi_n(0) =
    0.
\]

Daraus folgt der Ansatz
\[
    c_m = A \frac{\phi_m^*(0)}{E_m - E}
\]
analog zu \br7. Diesen Ansatz setze ich in die Gleichung ein und erhalte:
\[
    \sum_{n=0}^\infty \frac{\phi_n^*(0) \phi_n(0)}{E_n - E} = \frac{a_0}2.
\]

Als nächsten Schritt werden die Eigenfunktionen des harmonischen Oszillators
$\phi_n$ eingesetzt:
\[
    \sum_{n=0}^\infty \frac{1}{2^n n! \sqrt\piup} \frac{H_n^*(0) H_n(0)}{E_n -
    E} = \frac{a_0}2
\]

Mit
\[
    H_n(0) = \frac{2^n \sqrt\piup}{\Gamma\del{\frac{1-n}2}}
\]
folgt:
\[
    \sum_{n=0}^\infty \frac{2^n \sqrt\piup}{n!} \frac{1}{E_n -
    E} \Gamma\del{\frac{1-n}2}^{-2} = \frac{a_0}2
\]

$E_n$ einsetzen:
\[
    \sum_{n=0}^\infty \frac{2^n \sqrt\piup}{n!} \frac{1}{\frac 12 + n -
    E} \Gamma\del{\frac{1-n}2}^{-2} = \frac{a_0}2
\]

Mit Mathematica lässt sich diese Summe berechnen, wenn man nur über die geraden
$n$ summieren lässt:
\[
    \frac{2 \Gamma\del{\frac 14(5-2E)}}{(2E-1)\Gamma\del{\frac 14(3-2E)}}
\]

\begin{figure}[htbp]
    \centering
    \begin{tikzpicture}
        \begin{axis}[
                %grid=major,
                xmin=-5,
                xmax=5,
                xlabel=$1/a_0$,
                ylabel=$E$,
            ]
            \addplot[black] table {_build/gamma_data.txt};
        \end{axis}
    \end{tikzpicture}
    \caption{%
        %
    }
    \label{fig:}
\end{figure}

\subsection{Ohne Vorfaktor}

Ohne den Vorfaktor $2^n$ kann Mathematica die Reihe umformen:
\[
    \sum_{n=0}^\infty \frac{\sqrt\pi}{n!} \frac{1}{\frac 12 + n -
    E} \Gamma\del{\frac{1-n}2}^{-2} = - \frac{2}{(2E-1)\sqrt\piup}
    {_2F_1}\del{\frac 12, \frac 14(1-2E), \frac 14(5-2E), \frac 14}
\]

% vim: ft=tex spell spelllang=de tw=79
