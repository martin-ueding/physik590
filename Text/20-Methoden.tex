% Copyright © 2014 Martin Ueding <dev@martin-ueding.de>

\chapter{Methoden}

\section{Harmonischer Oszillator}

Der harmonische Oszillator ist ein elementares quantenmechanisches System, das
durch den Hamiltonoperator
\[
    \hat H = \frac{1}{2m} \hat p^2 + \frac{m}{2} \omega^2 \hat x^2
\]
bestimmt wird \parencite[(3.1)]{Schwabl/Quantenmechanik}. Die Energieeigenwerte
lassen sich durch Operatormethoden am elegantesten berechnen. Dazu führe ich
die dimensionslose Länge
\[
    \xi := \sqrt{\frac{m\omega}{\hbar}} x
\]
ein. Im Ortsraum ist $\hat p = - \iup \hbar \pd{}x$. Der dimensionslose Impuls
ist demnach $\hat \pi^2 = - \hbar\omega m \pd[2]{}\xi$. Somit kann der
Hamiltonoperator umgeschrieben werden zu:
\[
    \hat H = \frac{\hbar\omega}2 \del{\xi^2 - \dpd[2]{}\xi}.
\]

Mit den Leiteroperatoren
\[
    \hat a := \frac1{\sqrt2} \del{\hat \xi + \iup \hat \pi}
    \quad\text{und}\quad
    \hat a^\dagger := \frac1{\sqrt2} \del{\hat \xi - \iup \hat \pi}
\]
kann der Hamiltonoperator kompakt als
\[
    \hat H = \hbar\omega \cdot \del{\hat a^\dagger \hat a + \frac12}
\]
geschrieben werden \parencite[(3.8)]{Schwabl/Quantenmechanik}. Dabei habe ich
schon den Kommutator $[\hat a, \hat a^\dagger] = 1$ ausgenutzt. Der Operator
$\hat a^\dagger \hat a$ ist der Beseztungszahloperator $\hat n$. In der
Energieeigenbasis $\{ \ket n \}_{n \in \N_0}$ sind die Eigenwerte von $\hat n$
durch $n$ gegeben, so dass die Energieeigenwerte
\[
    E_n = \hbar\omega \cdot \del{n + \frac 12}
\]
gegeben sind.

\section{Pfadintegral}

% TODO Quellenangabe.
Feynmans Pfadintegralformalismus erlaubt es ein quantenmechanisches System mit
$n$ Freiheitsgraden in ein klassisches System mit $n+1$ Freiheitsgraden zu
überführen. Dieses System kann dann mit Methoden ähnlich der statistischen
Physik behandelt werden. Analog zur kanonischen Zustandssumme tritt hier eine
Summe über alle Weltlinien zwischen festem Anfangs- und Endpunkt auf
\parencite[(2.7)]{Creutz/Statistical_Approach_QM}:
\[
    Z(x_\text E, x_\text A) = \sum_{\text{Trajektorien $j$}} \exp\del{ \iup
    \frac{S_j}{\hbar}}.
\]

Die Phase mit der Wirkung $S$ wird stark variieren, so dass es ab dieser Stelle
hilfreich ist, imaginäre Zeit zu betrachten. Dazu wird $\tau := \iup t$
eingeführt. Schreibt man $\int [\dif x]$ für eine Integration über alle
Trajektorien $x(\tau)$, die $x(0) = x_\text A$ und $x(T) = x_\text E$ erfüllen,
so kann die Summe umgeschrieben werden zu
\parencite[(2.1)]{Creutz/Statistical_Approach_QM}:
\[
    \int [\dif x] \, \exp\del{-\frac{S[x]}\hbar}.
\]
Dieser Integrand wir nur in der Nähe der minimalen Wirkung Beiträge liefern.
Dies reduziert den Bereich des Phasenraums, über den integriert werden muss,
derart, dass die Integration mit Monte Carlo Methoden gut möglich ist.

Die Wirkung ist auch hier die Zeitintegration der Lagrangefunktion
\parencite[(2.5)]{Creutz/Statistical_Approach_QM}:
\[
    S = \int_0^T \dif \tau \, \del{\frac 12 m \del{\dpd x\tau(\tau)}^2 + V(x)}
\]

Diskretisiert man die Weltlinien, so wird es möglich, die Integration über alle
Weltlinien auszuführen. Das Integrationsmaß wird bei einer Unterteilung in
$N+1$ ($0, \ldots, N$)
Zeitpunkte zu:
\[
    \int [\dif x] \leadsto \prod_{j = 1}^{N-1} \int_{-\infty}^\infty \dif x_j.
\]
Der Abstand der diskreten Zeitpunkte ist der Einfachheit halber konstant und
wird als $a$ definiert. Somit gilt $T = Na$. Durch die Diskretisierung werden
Weltlinien unterdrückt, die sich zwischen den Zeitpunkten stark ändern. Im
Gegensatz zur reellen Zeit, in der Weltlinien mit belieber Wirkung in die
Zustandssumme einfließen können, fließen mit imaginärer Zeit Weltlinien mit
großer Geschwindigkeit (und daher großer Wirkung) deutlich weniger stark ein.
Daher ist die Benutzung eines Zeitgitters in Kombination mit imaginärer Zeit
legitim. Im Grenzfall $a \to 0$ erhält man den kontinuierlichen Fall zurück.

Durch die diskrete Zeit wird in der Wirkung die Zeitableitung durch einen
Differenzenquotient ersetzt. Die Lagrangefunktion hängt nur noch von zwei
aufeinanderfolgenden Koordinaten ab. Die Wirkung einer Weltlinie $\vec x_k$
wird dann gegeben durch
\parencite[(3.2)]{Creutz/Statistical_Approach_QM}:
\[
    S(\vec x_k) = a \sum_{j = 1}^N L\del{x_j^{(k)}, x_{j+1}^{(k)}},
\]
wobei periodische Randbedingungen mit $x_{N} = x_0$ festgesetzt wurden.

\section{Integration über den Phasenraum}

Der Raum der möglichen diskreten Weltlinien hat $N$ Dimensionen und ist
unendlich groß. Aufgrund der hohen Dimension sind Monte Carlo Methoden zu
bevorzugen. Der Bereich der Weltlinien, die einen nennenswerten Beitrag zum
Integral liefern, ist jedoch aufgrund der exponentiellen Gewichtung recht
klein. Der Algorithmus von Metropolis wählt die Weltlinien entsprechend dieser
Gewichtung so aus, dass ein ungewichteter Mittelwert dieser Weltlinien einen
Schätzwert für die gewünschte Größe gibt
\parencite[434]{Creutz/Statistical_Approach_QM}. Die Verteilung der
Trajektorien $\vec x_k$ wird durch
\begin{equation}
    \label{eq:p_x_k}
    p(\vec x_k) = \frac{\exp(-S(\vec x_k))}{\int [\dif x] \exp(-S(\vec x))}
\end{equation}
gegeben \parencite[(3.6)]{Creutz/Statistical_Approach_QM}. Werden $M$
Weltlinien durch den Metropolisalgorithmus generiert, so ist der Schätzwert
$\overline A$ für
eine Größe $\bracket A$
\parencite[(3.7)]{Creutz/Statistical_Approach_QM}:
\[
    \overline A = \frac1M \sum_{k=1}^M A(\vec x_k).
\]

\subsection{Markovketten}

Um die Weltlinien entsprechend gewichtet generieren zu können, kommt ein
Markovprozess zum Einsatz, der aus einer Weltlinie die nächste generiert. Für
$M \to \infty$ nähert sich die Verteilung der Benötigten an.
\parencite[434]{Creutz/Statistical_Approach_QM}

% Eigenschaften von Markovketten

Die Weltlinien des einen Oszillators, die aus den $N$ Koordinaten $\{x_j\}$
bestehen, können auch als Koordinaten für $N$ Teilchen interpretiert werden.
Somit ist eine Trajektorie ein Zustand dieses Vielteilchensystems. Man kann nun
eine Übergangsmatrix $W_{ij}$ definieren, die die Wahrscheinlichkeit angibt,
dass das System vom Zustand $i$ in den Zustand $j$ wechselt. Aufgrund der
diskreten Zeit entspricht ein Zeitschritt einem Übergangsschritt.

Die Matrixelemente müssen $W_{ij} \geq 0$, sowie muss $\sum_j W_{ij} = 1$
gelten, da jeder Zustand in einen nächsten (oder sich selbst) übergehen muss
\parencite[(3.8)]{Creutz/Statistical_Approach_QM}. Da die $x_j$ aus $\R$
stammen, muss die Übergangsmatrix eine Wahrscheinlichkeitsdichte sein. Die
analogen Bedingungen sind dann $W(x_i, x_j) \geq 0$ und $\int \dif x_j \,
W(x_i, x_j) = 1$ für alle $x_i$
\parencite[(3.9)]{Creutz/Statistical_Approach_QM}.

% Verkettung von Schritten
% Eigenvektor, Wahrscheinlichkeitsdichte
% Forderungen an W

Die Übergangswahrscheinlichkeit über einen Zwischenschritt wird durch
Integration über die Zwischenposition,
\[
    W^{(2)}(x_i, x_k) = \int \dif x_j \, W(x_i, x_j) W(x_j, x_k),
\]
ermittelt \parencite[(3.9)]{Creutz/Statistical_Approach_QM}. Daraus lässt die
Rekursionsformel
\[
    W^{(n)}(x_i, x_k) = \int \dif x_j \, W^{(n-1)}(x_i, x_j) W(x_j, x_k),
\]
konstruieren \parencite[(3.10)]{Creutz/Statistical_Approach_QM}. Aufgrund der
Eigenschaften von $W$ lässt sich zeigen\footnote{Siehe
    \parencite[435]{Creutz/Statistical_Approach_QM} und
\parencite[Anhang~B]{Creutz/Statistical_Approach_QM}.}, dass es eine
Zustandsverteilungsdichte $P$ gibt, so dass sie linker Eigenvektor von
$W^{(\infty)}$ ist. Das $W$, das für die Pfadintegrale benötigt wird, soll gerade
$p(x)$ aus \eqref{eq:p_x_k} als Eigenvektor haben. Somit sind die
Anforderungen an $W$ die folgenden \parencite[(3.18)]{Creutz/Statistical_Approach_QM}:
\begin{itemize}
    \item
        $W(x_i, x_j) > 0$ falls $p(x_i) > 0$ und $p(x_j) > 0$,
    \item
        $\int \dif x_j \, W(x_i, x_j) = 1$,
    \item
        $p(x_j) = \int \dif x_i \, p(x_i) W(x_i, x_j)$.
\end{itemize}

% Detailed Balance

Um die Bedingungen zu erfüllen, ist die Wahl von $W$ so, dass
\[
    \frac{W(x_i, x_j)}{W(x_j, x_i)} = \frac{p(x_j)}{p(x_i)}
\]
gilt, möglich \parencite[(3.23)]{Creutz/Statistical_Approach_QM}. In der Quelle
wird die Relation „detailed balance condition“ genannt.

Im Algorithmus geht nicht die ganze Trajektorie $\vec x$ in einem Schritt auf
eine neue Trajektorie $\vec x'$ über. Vielmehr werden die $x_j$ einzeln in ein
$x_j'$ überführt. Nach $N$ dieser Einzelschritte ist eine neue Weltlinie $\vec
x'$ erzeugt. Einer dieser Einzelschritte führt die Weltlinie $\vec x$ in eine
Weltlinie $\tilde{\vec x}$ über. Dabei wird nur die Koordinate $x_j$ zu $x_j'$
geändert. Mit \eqref{eq:p_x_k} kann man daher die Beziehung
\parencite[(3.25)]{Creutz/Statistical_Approach_QM}
\[
    \frac{W(\vec x, \tilde{\vec x})}{W(\tilde{\vec x}, \vec x)}
    = \frac{\exp(-S(x_j'))}{\exp(-S(x_j))},
\]
herleiten, welche durch die Wahl $W(x_j, x_j') \propto \exp(-S(x_j'))$ zu
erfüllen ist \parencite[(3.27)]{Creutz/Statistical_Approach_QM}. Dieser
Algorithmus wird „heat bath algorithm“ bezeichnet und hat den Nachteil, dass
es schwer ist, $x_j'$ entsprechend der Verteilung $\exp(-S(x_j'))$ zu ziehen
\parencite[438]{Creutz/Statistical_Approach_QM}.

% Metropolis Algorithmus

Der Algorithmus von Metropolis et al.\@ umgeht dieses Problem, indem er nach
einer einfach zu implementierenden Vorschrift, die $x_j'$ generiert
\parencite[439]{Creutz/Statistical_Approach_QM}:

\begin{algorithmic}
    \For{$j \gets 1, \ldots, N$}
        \State $x_j' \gets$ \Call{Zufallzahl}{$-\infty$, $\infty$}
        \State $\Deltaup S \gets S(x_0, \ldots, x_{j-1}, x_j',
        x_{j+1}, \ldots, x_N) - S(\{x_i\})$

        \If{$\Deltaup S \leq 0$}
            \State $x_j \gets x_j'$
        \Else
            \State $r \gets$ \Call{Zufallszahl}{0, 1}
            \If{$r < \exp(-\Deltaup S)$}
                \State $x_j \gets x_j'$
            \EndIf
        \EndIf
    \EndFor
\end{algorithmic}

Es werden also neue Werte für $x_j$, die $x_j'$ aus einer Verteilung gezogen.
Ist die Änderung der Wirkung $\Deltaup S$ negativ, so wird der neue Wert
übernommen. Andernfalls wird der Wert mit der Wahrscheinlichkeit
$\exp(-\Deltaup S)$ übernommen. Wenn der Abstand zwischen $x_j'$ und $x_j$ groß
ist, ist die Wahrscheinlichkeit, dass der Wert übernommen wird, sehr gering.
Daher ist es effizienter, $x_j'$ nicht aus dem Interval $(-\infty, \infty)$ zu
ziehen, sondern eine Breite $\Delta$ einzufühen
\parencite[439]{Creutz/Statistical_Approach_QM}. Die Autoren schlagen eine
Einheitsverteilung im Bereich $[x_j - \Delta, x_j + \Delta]$ vor. Ich benutze
in meiner Implementierung eine Normalverteilung mit einer Standardabweichung
$\Delta$. Dies führt zu noch höheren Akzeptanzraten für neue Werte, die die
Wirkung $S$ erhöhen. Um den Übergang zu einer neuen Konfiguration noch weiter
zu beschleunigen, kann man das Ziehen eines neuen $x_j'$ mehrfach für das
gleiche $j$ wiederholen. Dadurch ist es möglich, dass es sich stärker
verändert, als in einem einzigen Schritt. Die Anzahl dieser Wiederholungen
möchte ich, wie die Autoren, mit $\bar n$ bezeichnen.

\citeauthor{Creutz/Statistical_Approach_QM} geben
\[
    W(x_j, x_j') = \frac1{N_0} \del{
        \Theta(-\Deltaup S) + \exp(- \Deltaup S) \Theta(\Deltaup S)
        + \int \dif x' \, \del{
            1 - \exp(-\Deltaup S)
        } \Theta(\Deltaup S) \delta(x_j' - x_j)
    }
\]
als Wahrscheinlichkeitsdichte für diesen Algorithmus an
\parencite[(3.28)]{Creutz/Statistical_Approach_QM}, wobei ich $\Deltaup S :=
S(x_j') - S(x_j)$ definiert habe, $\Theta$ die Heaviside Stufenfunktion und
$N_0$ das Volumen des Konfigurationsraumes ist. Mit dieser Gleichung kann
nachvollzogen werden, dass sie $W$ die geforderten Bedingungen erfüllt.

%\nocite{Blossier/Eigenvalue}
%\nocite{Busch/Two_Cold}

%\nocite{Thijssen/Computational_Physics}

%\nocite{Young/Jackknife}
%\nocite{Oser/Bootstrap}

%\nocite{Stroustrup/C++4}

% vim: ft=tex spell spelllang=de tw=79
