% Copyright © 2014 Martin Ueding <dev@martin-ueding.de>

\chapter{Methoden}

\section{Harmonischer Oszillator}

Der harmonische Oszillator ist ein elementares quantenmechanisches System, das
durch den Hamiltonoperator
\[
    \hat H = \frac{1}{2m} \hat p^2 + \frac{m}{2} \omega^2 \hat x^2
\]
bestimmt wird \parencite[(3.1)]{Schwabl/Quantenmechanik}. Die Energieeigenwerte
lassen sich durch Operatormethoden am elegantesten berechnen. Dazu führe ich
die dimensionslose Länge
\[
    \xi := \sqrt{\frac{m\omega}{\hbar}} x
\]
ein. Im Ortsraum ist $\hat p = - \iup \hbar \pd{}x$. Der dimensionslose Impuls
ist demnach $\hat \pi^2 = - \hbar\omega m \pd[2]{}\xi$. Somit kann der
Hamiltonoperator umgeschrieben werden zu:
\[
    \hat H = \frac{\hbar\omega}2 \del{\xi^2 - \dpd[2]{}\xi}.
\]

Mit den Leiteroperatoren
\[
    \hat a := \frac1{\sqrt2} \del{\hat \xi + \iup \hat \pi}
    \quad\text{und}\quad
    \hat a^\dagger := \frac1{\sqrt2} \del{\hat \xi - \iup \hat \pi}
\]
kann der Hamiltonoperator kompakt als
\[
    \hat H = \hbar\omega \cdot \del{\hat a^\dagger \hat a + \frac12}
\]
geschrieben werden \parencite[(3.8)]{Schwabl/Quantenmechanik}. Dabei habe ich
schon den Kommutator $[\hat a, \hat a^\dagger] = 1$ ausgenutzt. Der Operator
$\hat a^\dagger \hat a$ ist der Beseztungszahloperator $\hat n$. In der
Energieeigenbasis $\{ \ket n \}_{n \in \N_0}$ sind die Eigenwerte von $\hat n$
durch $n$ gegeben, so dass die Energieeigenwerte
\[
    E_n = \hbar\omega \cdot \del{n + \frac 12}
\]
gegeben sind.

\section{Pfadintegral}

% TODO Quellenangabe.
Feynmans Pfadintegralformalismus erlaubt es ein quantenmechanisches System mit
$n$ Freiheitsgraden in ein klassisches System mit $n+1$ Freiheitsgraden zu
überführen. Dieses System kann dann mit Methoden ähnlich der statistischen
Physik behandelt werden. Analog zur kanonischen Zustandssumme tritt hier eine
Summe über alle Weltlinien zwischen festem Anfangs- und Endpunkt auf
\parencite[(2.7)]{Creutz/Statistical_Approach_QM}:
\[
    Z(x_\text E, x_\text A) = \sum_{\text{Trajektorien $j$}} \exp\del{ \iup
    \frac{S_j}{\hbar}}.
\]

Die Phase mit der Wirkung $S$ wird stark variieren, so dass es ab dieser Stelle
hilfreich ist, imaginäre Zeit zu betrachten. Dazu wird $\tau := \iup t$
eingeführt. Schreibt man $\int [\dif x]$ für eine Integration über alle
Trajektorien $x(\tau)$, die $x(0) = x_\text A$ und $x(T) = x_\text E$ erfüllen,
so kann die Summe umgeschrieben werden zu:
\parencite[(2.1)]{Creutz/Statistical_Approach_QM}:
\[
    \int [\dif x] \, \exp\del{-\frac{S[x]}\hbar}.
\]
Dieser Integrand wir nur in der Nähe der minimalen Wirkung Beiträge liefern.
Dies reduziert den Bereich des Phasenraums, über den integriert werden muss,
derart, dass die Integration mit Monte Carlo Methoden gut möglich ist.

Die Wirkung ist auch hier die Zeitintegration der Lagrangefunktion
\parencite[(2.5)]{Creutz/Statistical_Approach_QM}:
\[
    S = \int_0^T \dif \tau \, \del{\frac 12 m \del{\dpd x\tau(\tau)}^2 + V(x)}
\]

Diskretisiert man die Weltlinien, so wird es möglich, die Integration über alle
Weltlinien auszuführen. Das Integrationsmaß wird bei einer Unterteilung in
$N+1$ ($0, \ldots, N$)
Zeitpunkte zu:
\[
    [\dif x] \leadsto \prod_{j = 1}^{N-1} \dif x_j.
\]
Der Abstand der diskreten Zeitpunkte ist der Einfachheit halber konstant und
wird als $a$ definiert. Somit gilt $T = Na$. Durch die Diskretisierung werden
Weltlinien unterdrückt, die sich zwischen den Zeitpunkten stark ändern. Im
Gegensatz zur reellen Zeit, in der Weltlinien mit belieber Wirkung in die
Zustandssumme einfließen können, fließen mit imaginärer Zeit Weltlinien mit
großer Geschwindigkeit (und daher großer Wirkung) deutlich weniger stark ein.
Daher ist die Benutzung eines Zeitgitters in Kombination mit imaginärer Zeit
legitim. Im Grenzfall $a \to 0$ erhält man den kontinuierlichen Fall zurück.

%\nocite{Blossier/Eigenvalue}
%\nocite{Busch/Two_Cold}

%\nocite{Thijssen/Computational_Physics}

%\nocite{Young/Jackknife}
%\nocite{Oser/Bootstrap}

%\nocite{Stroustrup/C++4}

% vim: ft=tex spell spelllang=de tw=79
