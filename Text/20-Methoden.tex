% Copyright © 2014 Martin Ueding <dev@martin-ueding.de>

\chapter{Methoden}

\section{Harmonischer Oszillator}

Der harmonische Oszillator ist ein elementares quantenmechanisches System, das
durch den Hamiltonoperator
\[
    \hat H = \frac{1}{2m} \hat p^2 + \frac{m}{2} \omega^2 \hat x^2
\]
bestimmt wird \parencite[(3.1)]{Schwabl/Quantenmechanik}. Die Energieeigenwerte
lassen sich durch Operatormethoden am elegantesten berechnen. Dazu führe ich
die dimensionslose Länge
\[
    \xi := \sqrt{\frac{m\omega}{\hbar}} x
\]
ein. Im Ortsraum ist $\hat p = - \iup \hbar \pd{}x$. Der dimensionslose Impuls
ist demnach $\hat \pi^2 = - \hbar\omega m \pd[2]{}\xi$. Somit kann der
Hamiltonoperator umgeschrieben werden zu:
\[
    \hat H = \frac{\hbar\omega}2 \del{\xi^2 - \dpd[2]{}\xi}.
\]

Mit den Leiteroperatoren
\[
    \hat a := \frac1{\sqrt2} \del{\hat \xi + \iup \hat \pi}
    \quad\text{und}\quad
    \hat a^\dagger := \frac1{\sqrt2} \del{\hat \xi - \iup \hat \pi}
\]
kann der Hamiltonoperator kompakt als
\[
    \hat H = \hbar\omega \del{\hat a \hat a^\dagger + \frac12}
\]
geschrieben werden \parencite[(3.8)]{Schwabl/Quantenmechanik}. Dabei habe ich
schon den Kommutator $[\hat a, \hat a^\dagger] = 1$ ausgenutzt.

% vim: ft=tex spell spelllang=de tw=79
