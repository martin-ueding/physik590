% Copyright © 2014 Martin Ueding <dev@martin-ueding.de>

\chapter{Herleitungen}

\section{Eindimensionale Wellenfunktion mit $\delta$-Störung}

% TODO Einleitung für diesen Teil schreiben.

\newcommand\br[1]{\parencite[(#1)]{Busch/Two_Cold}}

In \citetitle{Busch/Two_Cold} leiten die Autoren eine analytische Lösung für
einen anharmonischen Oszillator her, bei welchem der Störterm eine
Deltadistribution ist. Damit berechnen sie dann Wellenfunktionen und
Energiewerte.

Die Herleitung für den eindimensionalen Fall wird als analog zum
dreidimensionalen Fall beschrieben. Anstelle des dreidimensionalen Potentials
$\sqrt 2 \piup a_0 \delta^{(3)}(\vec r) \pd{}r r$ soll einfach nur $V(x_1 -
x_2) = - 2/a_0 \delta(x_1 - x_2)$ benutzt werden
\parencite[Fußnote~20]{Busch/Two_Cold}.

Ich beginne meine Herleitung nach der Separation von Schwerpunkts- und
Relativbewegung analog zu \br2. Der Hamiltonoperator für den anharmonischen
Oszillator ist
\[
    H_\text{rel} = \underbrace{- \frac12 \nabla_x^2 + \frac 12
    x^2}_{H_\text{osz}} - \frac2{a_0} \delta(x).
\]

Dieser Operator löst (analog \br3) die zeitunabhängige Schrödingergleichung
\begin{equation}
    \label{eq:Bus_3}
    \del{H_\text{osz} - \frac2{a_0} \delta(x)} \Psi(x) = E \Psi(x).
\end{equation}

Die Autoren entwickeln die unbekannte Wellenfunktion $\Psi$ nun nach den
bekannten Wellenfunktionen des harmonischen Oszillators \br4:
\[
    \Psi(x) = \sum_{n=0}^\infty c_n \phi_n(x).
\]

Im eindimensionalen Fall gibt es keinen Drehimpuls $l$. Aber $\phi_n$ mit
ungeradem $n$ tragen nicht bei, da diese bei $x = 0$ einen Nulldurchgang haben.
Somit bleiben in der Summe nur die geraden $n$.


Dies setze ich (analog zu \br5) in die Schrödingergleichung \eqref{eq:Bus_3}
ein:
\[
    \del{H_\text{osz} - E - \frac2{a_0} \delta(x)}
    \sum_{n=0}^\infty c_n \phi_n(x)
    = 0
    \iff
    \sum_{n=0}^\infty (E_n - E) c_n \phi_n(x)
    \frac2{a_0} \delta(x)
    \sum_{n=0}^\infty c_n \phi_n(x)
    = 0
\]

Dies wird nun auf $\phi_m^*$ projiziert, i.\,e. Multiplikation von rechts und
Integration über $x$:
\[
    \int_{-\infty}^\infty \dif x \, \sum_{n=0}^\infty
    (E_n - E) c_n \phi_m^*(x) \phi_n(x)
    - 
    \int_{-\infty}^\infty \dif x \,  \frac2{a_0} \phi_m^*(x) \delta(x)
    \sum_{n=0}^\infty c_n \phi_n(x) = 0
\]

Mit der Orthogonalitätsrelation für die Eigenfunktionen
\[
    \int_{-\infty}^\infty \dif x \phi_m^*(x) \phi_n(x) = \braket{m|n} =
    \delta_{mn}
\]
folgt für die Schrödingergleichung
\[
    (E_m - E) c_m - \frac2{a_0} \phi_m^*(0) \sum_{n=0}^\infty c_n \phi_n(0) =
    0.
\]

Über den Index $n$ wird summiert, dieser ist außerdem für alle $m$ gleich.
Somit kann dieser Teil in eine Konstante $A$ verschoben werden. Daraus folgt
der Ansatz
\[
    c_m = A \frac{\phi_m^*(0)}{E_m - E}
\]
analog zu \br7. Diesen Ansatz setze ich in die Gleichung ein und erhalte:
\[
    \sum_{n=0}^\infty \frac{\phi_n^*(0) \phi_n(0)}{E_n - E} = \frac{a_0}2.
\]

Als nächsten Schritt werden die Eigenfunktionen des harmonischen Oszillators
$\phi_n$ eingesetzt. An der Stelle $x = 0$ verschwindet der Exponentialterm:
\[
    \sum_{n=0}^\infty \frac{1}{2^n n! \sqrt\piup} \frac{H_n^*(0) H_n(0)}{E_n -
    E} = \frac{a_0}2
\]

Mit dem Wert der Hermitepolynome am Ursprung,
\[
    H_n(0) = \frac{2^n \sqrt\piup}{\Gamma\del{\frac{1-n}2}},
\]
folgt:
\[
    \sum_{n=0}^\infty \frac{2^n \sqrt\piup}{n!} \frac{1}{E_n -
    E} \Gamma\del{\frac{1-n}2}^{-2} = \frac{a_0}2
\]

In die obige Gleichung setze ich die bekannten Energiewerte $E_n = n + 1/2$ für
den harmonischen Oszillator ein:
\[
    \sum_{n=0}^\infty \frac{2^n \sqrt\piup}{n!} \frac{1}{\frac 12 + n -
    E} \Gamma\del{\frac{1-n}2}^{-2} = \frac{a_0}2
\]

Mit Mathematica lässt sich diese Summe berechnen, wenn man nur über die geraden
$n$ summieren lässt:
\[
    \frac{2 \Gamma\del{\frac 14(5-2E)}}{(2E-1)\Gamma\del{\frac 14(3-2E)}}
    = \frac{a_0}2.
\]
Ausgedrückt mit der Kopplungsstärke $1/a_0$ ist die Relation
\[
    \frac{(2E-1)\Gamma\del{\frac 14(3-2E)}}{4 \Gamma\del{\frac 14(5-2E)}}
    = \frac1{a_0}.
\]


Dies ist in Abbildung~\ref{fig:E_a0} dargestellt.

\begin{figure}[htbp]
    \centering
    \tikzsetnextfilename{E_a0}
    \begin{tikzpicture}
        \begin{axis}[
                %grid=major,
                xmin=-5,
                xmax=5,
                xlabel=$1/a_0$,
                ylabel=$E$,
                minor y tick num=4,
            ]
            \addplot[black] table {_build/gamma_data.txt};
        \end{axis}
    \end{tikzpicture}
    \caption{%
        Energie $E$ in Abhängigkeit von der Kopplungsstärke $1/a_0$.
    }
    \label{fig:E_a0}
\end{figure}

% vim: ft=tex spell spelllang=de tw=79
