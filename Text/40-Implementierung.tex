% Copyright © 2014 Martin Ueding <dev@martin-ueding.de>

\chapter{Implementierung}

Mein Programm ist freie Software und unter
\url{http://martin-ueding.de/de/university/physik590/} erhältlich.

Die Weltlinie wird in der Regel anfangs mit zufälligen Werten für die $x_j$ aus
dem Interval $(-\initialrandomwidth, \initialrandomwidth)$ initialisiert. Bevor
Werte aufgenommen werden, durchläuft das System $\preiterations = \num{50}$
Monte Carlo Iterationen zur Relaxation. Danach sieht die Weltlinie wie in
Abbildung~\ref{fig:relaxiert} aus.

\begin{figure}[htbp]
    \centering
    \begin{tikzpicture}
        \begin{axis}[
                width=\linewidth,
                height=0.5\linewidth,
                xlabel={$j = t/a$},
                ylabel={$x_j = x(t/a)$},
            ]
            \addplot[black] table {11EA8F-trajectory-04-more_iterations.csv};
        \end{axis}
    \end{tikzpicture}
    \caption{%
        Weltlinie nach $\preiterations = \num{50}$ Iterationen zur Relaxation.
        Dabei sind die Parameter $\timestep = \num{0.1}$, $\initialrandomwidth
        = \margin = \num{0.632456}$, und $\mu^2 = \num{1}$.
    }
    \label{fig:relaxiert}
\end{figure}

\section{Harmonischer Oszillator}

\subsection{Aufenthaltswahrscheinlichkeit}



Abbildung~\ref{fig:histogram_01}
Abbildung~\ref{fig:histogram_10}

\begin{figure}[htbp]
    \centering
    \begin{tikzpicture}
        \begin{axis}[
                width=\linewidth,
                height=0.55\linewidth,
                xlabel={$x$},
                ylabel={$|\psi(x)|^2$},
                grid=major,
                xmin=-3,
                xmax=3,
                legend entries={
                    {Simulation},
                    {Theorie},
                    Kontinuum,
                },
            ]
            \addplot[
                black,
                only marks,
                mark=|,
                error bars/.cd,
                y dir=both,
                y explicit
            ] table[y error index=2] {histogram-korrigiert.csv};
            \addplot[black] table {_build/histogram_theory_01.csv};
            \addplot[black, dashed] table {_build/histogram_theory_continous.csv};
        \end{axis}
    \end{tikzpicture}
    \caption{%
        Aufenthaltswahrscheinlichkeit. $\timestep =
        \num{0.1}$, $\Delta = \num{0.632456}$, $\mu^2 = \num{1}$, $\timesites =
        \num{1000}$, $\iterations = \num{1000}$, $\bootstrapsamples = \num{10000}$.
    }
    \label{fig:histogram_01}
\end{figure}

\begin{figure}[htbp]
    \centering
    \begin{tikzpicture}
        \begin{axis}[
                width=\linewidth,
                height=0.55\linewidth,
                xlabel={$x$},
                ylabel={$|\psi(x)|^2$},
                grid=major,
                xmin=-3,
                xmax=3,
                legend entries={
                    {Simulation},
                    {Theorie},
                    Kontinuum,
                },
            ]
            \addplot[
                black,
                only marks,
                mark=|,
                error bars/.cd,
                y dir=both,
                y explicit
            ] table[y error index=2] {DC5256-histogram-position-resultset.csv};
            \addplot[black] table {_build/histogram_theory_1.csv};
            \addplot[black, dashed] table {_build/histogram_theory_continous.csv};
        \end{axis}
    \end{tikzpicture}
    \caption{%
        Aufenthaltswahrscheinlichkeit. $\timestep =
        \num{1}$, $\margin = \num{0.632456}$, $\mu^2 = \num{1}$, $\timesites =
        \num{1000}$, $\iterations = \num{5000}$, $\bootstrapsamples = \num{1000}$.
    }
    \label{fig:histogram_10}
\end{figure}

\subsection{Korrelationen}

\subsection{Energiewerte}

\section{Anharmonischer Oszillator}

\begin{figure}[htbp]
    \centering
    \begin{tikzpicture}
        \begin{axis}[
                width=\linewidth,
                height=0.55\linewidth,
                xlabel={$x$},
                ylabel={$|\psi(x)|^2$},
                grid=major,
            ]
            \addplot[
                black,
                only marks,
                mark=|,
                error bars/.cd,
                y dir=both,
                y explicit
            ] table[y error index=2] {FF7AA0-histogram-position-resultset.csv};
        \end{axis}
    \end{tikzpicture}
    \caption{%
        Aufenthaltswahrscheinlichkeit. $\timestep = \num{0.1}$,
        $\initialrandomwidth = \num{0}$, $\margin = \num{0.632456}$, $\mu^2 =
        \num{1}$, $\timesites = \num{1000}$, $\iterations = \num{10000}$,
        $\bootstrapsamples = \num{1000}$, $\gaussheight = \num{10}$,
        $\gausswidth = \num{1.0}$.
    }
    \label{fig:histogram_gauss}
\end{figure}


% vim: ft=tex spell spelllang=de tw=79
