% Copyright © 2014 Martin Ueding <dev@martin-ueding.de>

\chapter{Implementierung}

\section{Harmonischer Oszillator}


\begin{figure}[htbp]
    \centering
    \begin{tikzpicture}
        \begin{axis}[
                width=\linewidth,
                height=0.5\linewidth,
                xlabel={$j = t/a$},
                ylabel={$x_j = x(t/a)$},
            ]
            \addplot[black] table
            {trajectory-04-more_iterations-i{300}-o{1}-a{0.1}-s{0}-c{0}-margin{0.632456}.csv};
        \end{axis}
    \end{tikzpicture}
    \caption{%
        Weltlinie nach \num{300} Iterationen. Dabei sind die Parameter $a =
        \num{0.1}$, $\Delta = \num{0.632456}$, und $\mu^2 = \num{1}$.
    }
    \label{fig:}
\end{figure}

\subsection{Aufenthaltswahrscheinlichkeit}

\begin{figure}[htbp]
    \centering
    \begin{tikzpicture}
        \begin{axis}[
                width=\linewidth,
                height=0.55\linewidth,
                xlabel={$x$},
                ylabel={$|\psi(x)|^2$},
                grid=major,
                xmin=-3,
                xmax=3,
                legend entries={
                    {Simulation},
                    {Theorie},
                    Kontinuum,
                },
            ]
            \addplot[
                black,
                only marks,
                mark=|,
                error bars/.cd,
                y dir=both,
                y explicit
            ] table[y error index=2] {histogram-korrigiert.csv};
            \addplot[black] table {_build/histogram_theory_01.csv};
            \addplot[black, dashed] table {_build/histogram_theory_continous.csv};
        \end{axis}
    \end{tikzpicture}
    \caption{%
        Aufenthaltswahrscheinlichkeit. $a =
        \num{0.1}$, $\Delta = \num{0.632456}$, $\mu^2 = \num{1}$, $N =
        \num{1000}$, $M = \num{1000}$, $N_\text P = \num{10000}$.
    }
    \label{fig:histogram_01}
\end{figure}

\begin{figure}[htbp]
    \centering
    \begin{tikzpicture}
        \begin{axis}[
                width=\linewidth,
                height=0.55\linewidth,
                xlabel={$x$},
                ylabel={$|\psi(x)|^2$},
                grid=major,
                xmin=-3,
                xmax=3,
                legend entries={
                    {Simulation},
                    {Theorie},
                    Kontinuum,
                },
            ]
            \addplot[
                black,
                only marks,
                mark=|,
                error bars/.cd,
                y dir=both,
                y explicit
            ] table[y error index=2] {DC5256-histogram-position-resultset.csv};
            \addplot[black] table {_build/histogram_theory_1.csv};
            \addplot[black, dashed] table {_build/histogram_theory_continous.csv};
        \end{axis}
    \end{tikzpicture}
    \caption{%
        Aufenthaltswahrscheinlichkeit. $a =
        \num{1}$, $\Delta = \num{0.632456}$, $\mu^2 = \num{1}$, $N =
        \num{1000}$, $M = \num{5000}$, $N_\text P = \num{1000}$.
    }
    \label{fig:histogram_10}
\end{figure}

\subsection{Korrelationen}

\subsection{Energiewerte}

\section{Anharmonischer Oszillator}


% vim: ft=tex spell spelllang=de tw=79
