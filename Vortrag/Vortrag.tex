\documentclass[ngerman, fleqn]{beamer}

\usetheme{Dresden}
%\usetheme{Marburg}

\usepackage{header}

\usepackage{tikz}
\usepackage{pgfplots}
\pgfplotsset{
    compat=1.9,
    width=0.8\linewidth,
    xticklabel style={/pgf/number format/use comma},
    yticklabel style={/pgf/number format/use comma},
}
\usepgfplotslibrary{external}
\tikzexternalize[mode=list and make]
\tikzsetexternalprefix{_build/Abbildung-}

\usepackage{algpseudocode}

\newcommand\bootstrapsamples{N_\text P}
\newcommand\gausswidth{\sigma}
\newcommand\initialrandomwidth{\tilde \Delta}
\newcommand\iterations{M}
\newcommand\margin{\Delta}
\newcommand\mass{\mu}
\newcommand\preiterations{M'}
\newcommand\rounds{\bar n}
\newcommand\timesites{N}
\newcommand\timestep{a}

\newcommand\funcmeasure{\mathcal D \vec x}

\renewcommand\iup{\text i}
\renewcommand\eup{\text e}

\hypersetup{
    pdftitle={%
        Harmonischer Oszillator in der Pfadintegralformulierung in Euklidischer
        Raum-Zeit mit gitterfeldtheoretischen Methoden
    }
}

\title{%
    Gitterfeldtheoretische Behandlung\\
    des Harmonischen Oszillators\\
    in der Pfadintegralformulierung\\
    in Euklidischer Raum-Zeit
}
\subtitle{Präsentation zur Bachelorarbeit}
\author{Martin Ueding – mu@martin-ueding.de}
\date{Juli 2014}

\begin{document}

\begin{frame}
    \titlepage
\end{frame}

\begin{frame}
    \frametitle{Einleitung}

    Modelle in der Physik

    \begin{itemize}
        \item
            analytisch lösbar
        \item
            Störungstheorie
        \item
            andere Näherungsverfahren
        \item
            Numerik
    \end{itemize}
\end{frame}

\begin{frame}
    \tableofcontents
\end{frame}

\section{Pfadintegrale}

\subsection{Einleitung}

\begin{frame}
    \frametitle{Pfadintegral}

    Betrachte Trajektorien mit $x(0) = x_\text A$ und $x(T) = x_\text E$.

    Zustandssumme:
    \[
        Z(x_\text E, x_\text A) = \sum_{\text{Trajektorien $j$}} \exp(\iup S_j)
    \]

    Probleme:
    \begin{itemize}
        \item Summand oszilliert stark
        \item Trajektorien sind kontinuierlich
    \end{itemize}
\end{frame}

\begin{frame}
    \frametitle{Imaginäre Zeit}

    Wickrotation: $\tau := \iup t$

    Wirkung des Oszillators:
    \[
        S = \int_0^T \dif \tau \, \sbr{\frac 12 m \dot x^2 + V(x)}.
    \]
\end{frame}

\begin{frame}
    \frametitle{Diskretisierung}

    Summe muss Integral sein:
    \(
        \int \funcmeasure \, \exp\del{-S(x)}.
    \)

    Zeitgitter:
    \begin{itemize}
        \item
            diskrete Zeit: $\tau_j = a j$, $T = aN$
        \item
            $x(\tau) \to x_j$ mit $j = 0, \ldots, N$
        \item
            ersetze $\dot x$ durch Differenzenquotient
        \item
            \(
            \int \funcmeasure
            \to
            \prod_{j = 1}^{\timesites-1} \int_{-\infty}^\infty \dif x_j.
            \)
    \end{itemize}

\end{frame}

\subsection{Integration}

\begin{frame}
    \frametitle{Monte Carlo}

    \begin{itemize}
        \item 
            Integral über $N \gg 100$ Dimensionen
        \item
            Monte-Carlo-Methoden effizient
        \item
            Integrationsgebiet durch Gewichtung klein
    \end{itemize}

    Ziel: Trajektorien entsprechend $\exp(-S)$ generieren
\end{frame}

\begin{frame}
    \frametitle{Markovprozess}

    Gewünschte Verteilung:
    \[
        p\del{\vec x^{(k)}} = \frac{\exp\del{-S\del{\vec x^{(k)}}}}{\int
            \funcmeasure
        \, \exp\del{-S(\vec x)}}
    \]

    detailiertes Gleichgewicht:
    \[
        \frac{W\del{\vec x^{(i)}, \vec x^{(j)}}}{W\del{\vec x^{(j)}, \vec x^{(i)}}}
        = \frac{p\del{\vec x^{(j)}}}{p\del{\vec x^{(i)}}}
    \]
\end{frame}

\begin{frame}
    \frametitle{Metropolisalgorithmus}
    
    \begin{algorithmic}
        \For{$j \gets 1, \ldots, \timesites$}
            \State $x_j' \gets$ \Call{Zufallszahl}{$-\infty$, $\infty$}
            \State $\Deltaup S \gets S(x_0, \ldots, x_{j-1}, x_j',
            x_{j+1}, \ldots, x_\timesites) - S(\{x_i\})$

            \If{$\Deltaup S \leq 0$}
                \State $x_j \gets x_j'$
            \Else
                \State $r \gets$ \Call{Zufallszahl}{0, 1}
                \If{$r < \exp(-\Deltaup S)$}
                    \State $x_j \gets x_j'$
                \EndIf
            \EndIf
        \EndFor
    \end{algorithmic}
\end{frame}

\begin{frame}
    \frametitle{Modifikationen}

    \begin{itemize}
        \item
            Ziehen aus Normalverteilung um $x_j$

        \item
            wiederholtes Ziehen

        \item
            Auslassen von Trajektorien

        \item
            $\Deltaup S$ effizient berechnen
    \end{itemize}
\end{frame}

\section{Energiewerte}
\subsection{Virialsatz, Korrelationen und GEVP}

\begin{frame}
    \frametitle{Virialsatz}

    \begin{itemize}
    
        \item
    Allgemein: $\bracket{\vec F \vec r} + 2 \bracket{T} = 0$

\item
    Mit $\vec F = - \vnabla V$ folgt $2 T = m \bracket{v^2} = \bracket{x V'(x)}$

\item
    Hier: $V(x) = \mu^2 x^2 /2$

\item
    $E_0$ aus verfügbaren Werten:
    \[
        \bracket E = \bracket T + \bracket V
        = \frac12 \mu^2 \bracket{x^2} + \frac12 \mu^2 \bracket{x^2}
        = \mu^2 \bracket{x^2}
    \]
    \end{itemize}
\end{frame}

\begin{frame}
    \frametitle{Korrelationen}

    $E_1$ durch Korrelationen und $E_0$:
    \[
        E_1 = \lim_{\tau \to \infty} \frac{-1}{\Deltaup \tau} \ln
        \del{\frac{\Bracket{x(0) \, x(\tau + \Deltaup \tau)}}{\Bracket{x(0) \,
        x(\tau)}}} + E_0
    \]

    Äquivalent zu
    \[
        f(\tau) := c_1 \exp\del{-\Deltaup E_1 \tau}.
    \]
\end{frame}


\begin{frame}
    \frametitle{Choleskyzerlegung}

    \begin{itemize}
        \item
            $A \vec x = \lambda B \vec x$

        \item
            Choleskyzerlegung $B = L L^\dagger$ 

        \item
            Umformungen
            \[
                L^{-1} A [L^\dagger]^{-1} \; L^\dagger \vec x
                =
                \lambda [L^\dagger \vec x].
            \]
            \[
                \tilde{A} := L^{-1} A [L^\dagger]^{-1}
                \eqnsep
                \tilde{\vec x} := L^\dagger \vec x,
            \]

        \item
            normales Eigenwertproblem
            \[
                \tilde{A} \tilde{\vec x} = \lambda \tilde{\vec x}.
            \]
    \end{itemize}
\end{frame}

\subsection{Eindimensionale Wellenfunktion mit $\delta$-Störung}

\begin{frame}
    Leer
\end{frame}

\section{Ergebnisse}

\subsection{Harmonischer Oszillator}

\begin{frame}
    \frametitle{Ungerade Eigenwerte}
    \tikzsetnextfilename{ED7AA2-eigenwerte-ungerade}
    \begin{tikzpicture}
        \begin{semilogyaxis}[
                width=\linewidth,
                height=0.55\linewidth,
                xlabel={$\tau$},
                ylabel={$\lambda_n(\tau)$},
                grid=major,
                legend entries={
                    {$n = 1$},
                    {$n = 3$},
                    {$n = 5$},
                },
            ]
            \addplot[black!100!white] table {../Text/CSV/lambdas/ED7AA2-eigenvalue-01-fit.csv};
            \addplot[black!60!white] table {../Text/CSV/lambdas/ED7AA2-eigenvalue-03-fit.csv};
            \addplot[black!30!white] table {../Text/CSV/lambdas/ED7AA2-eigenvalue-05-fit.csv};
            \addplot[
                black!100!white,
                only marks,
                mark=|,
                error bars/.cd,
                y dir=both,
                y explicit
            ] table[y error index=2] {../Text/CSV/lambdas/ED7AA2-eigenvalue-01.csv};

            \addplot[
                black!60!white,
                only marks,
                mark=|,
                error bars/.cd,
                y dir=both,
                y explicit
            ] table[y error index=2] {../Text/CSV/lambdas/ED7AA2-eigenvalue-03.csv};
            \addplot[
                black!30!white,
                only marks,
                mark=|,
                error bars/.cd,
                y dir=both,
                y explicit
            ] table[y error index=2] {../Text/CSV/lambdas/ED7AA2-eigenvalue-05.csv};
        \end{semilogyaxis}
    \end{tikzpicture}
\end{frame}

\subsection{Anharmonischer Oszillator}

%\begin{frame}
%    \printbibliography
%\end{frame}

\end{document}

% vim: spell spelllang=de
