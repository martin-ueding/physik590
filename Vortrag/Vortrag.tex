\documentclass[ngerman, fleqn]{beamer}

\usetheme{Dresden}
%\usetheme{Marburg}

\usepackage{header}

\usepackage{booktabs}

\usepackage{tikz}
\usepackage{pgfplots}
\pgfplotsset{
    compat=1.9,
    width=0.8\linewidth,
    xticklabel style={/pgf/number format/use comma},
    yticklabel style={/pgf/number format/use comma},
}
\usepgfplotslibrary{external}
\tikzexternalize[mode=list and make]
\tikzsetexternalprefix{_build/Abbildung-}


\pgfplotsset{
    colormap={whiteblack}{gray=(0.95) gray=(0)},
    max space between ticks=50pt,
    myerrorbar/.style={
        only marks,
        error bars/y dir=both,
        error bars/y explicit,
        mark options={scale=1, thick},
    },
}


\usepackage{algpseudocode}

\newcommand\bootstrapsamples{N_\text P}
\newcommand\gausswidth{\sigma}
\newcommand\initialrandomwidth{\tilde \Delta}
\newcommand\iterations{M}
\newcommand\margin{\Delta}
\newcommand\mass{\mu}
\newcommand\preiterations{M'}
\newcommand\rounds{\bar n}
\newcommand\timesites{N}
\newcommand\timestep{a}

\newcommand\funcmeasure{\mathcal D \vec x}

\renewcommand\iup{\text i}
\renewcommand\eup{\text e}

\hypersetup{
    pdftitle={%
        Gitterfeldtheoretische Behandlung des harmonischen Oszillators in der
        Pfadintegralformulierung in Euklidischer Raum-Zeit
    }
}

\definecolor{Set1_1}{rgb}{0.894118,0.101961,0.109804}
\definecolor{Set1_2}{rgb}{0.215686,0.494118,0.721569}
\definecolor{Set1_3}{rgb}{0.301961,0.686275,0.290196}
\definecolor{Set1_4}{rgb}{0.596078,0.305882,0.639216}
\definecolor{Set1_5}{rgb}{1.0,0.498039,0.0}
\definecolor{Set1_6}{rgb}{1.0,1.0,0.2}
\definecolor{Set1_7}{rgb}{0.65098,0.337255,0.156863}
\definecolor{Set1_8}{rgb}{0.968627,0.505882,0.74902}
\definecolor{Set1_9}{rgb}{0.6,0.6,0.6}

\title{%
    Gitterfeldtheoretische Behandlung\\
    des harmonischen Oszillators\\
    in der Pfadintegralformulierung\\
    in Euklidischer Raum-Zeit
}
\subtitle{Präsentation zur Bachelorarbeit}
\author{Martin Ueding – mu@martin-ueding.de}
\date{August 2014}

\begin{document}

\begin{frame}
    \titlepage
\end{frame}

\section*{Einleitung}

\begin{frame}
    \frametitle{Einleitung}

    Modelle in der Physik

    \begin{itemize}
        \item
            analytisch lösbar
        \item
            Störungstheorie
        \item
            andere Näherungsverfahren
        \item
            Numerik
    \end{itemize}
\end{frame}

\begin{frame}
    \tableofcontents
\end{frame}

\section{Pfadintegrale}

\frame\sectionpage

\subsection{Einleitung}

\begin{frame}
    \frametitle{Pfadintegral}

    Betrachte Trajektorien mit $x(0) = x_\text A$ und $x(T) = x_\text E$.

    Zustandssumme:
    \[
        Z(x_\text E, x_\text A) = \sum_{\text{Trajektorien $j$}} \exp(\iup S_j)
    \]

    Probleme:
    \begin{itemize}
        \item Summand oszilliert stark
        \item Trajektorien sind kontinuierlich
    \end{itemize}
\end{frame}

\begin{frame}
    \frametitle{Imaginäre Zeit}

    Wickrotation: $\tau := \iup t$

    Wirkung des Oszillators:
    \[
        S = \int_0^T \dif \tau \, \sbr{\frac 12 m \dot x^2 + V(x)}.
    \]

    Zustandssumme:
    \[
        Z(x_\text E, x_\text A) = \sum_{\text{Trajektorien $j$}} \exp(- S_j)
    \]
\end{frame}

\begin{frame}
    \frametitle{Diskretisierung}

    Summe muss Integral sein:
    \(
        \int \funcmeasure \, \exp\del{-S(x)}.
    \)

    Zeitgitter:
    \begin{itemize}
        \item
            diskrete Zeit: $\tau_j = a j$, $T = aN$
        \item
            $x(\tau) \to x_j$ mit $j = 0, \ldots, N$
        \item
            ersetze $\dot x$ durch Differenzenquotient
        \item
            \(
            \int \funcmeasure
            \to
            \prod_{j = 1}^{\timesites-1} \int_{-\infty}^\infty \dif x_j.
            \)
    \end{itemize}

\end{frame}

\subsection{Integration über den Phasenraum}

\frame\subsectionpage

\begin{frame}
    \frametitle{Monte Carlo}

    \begin{itemize}
        \item 
            Integral über $N \gg 100$ Dimensionen
        \item
            Monte-Carlo-Methoden effizient
        \item
            Gebiet mit Beiträgen klein
    \end{itemize}

    Ziel: Trajektorien entsprechend $\exp(-S)$ generieren
\end{frame}

\begin{frame}
    \frametitle{Markovprozess}

    Gewünschte Verteilung:
    \[
        p\del{\vec x^{(k)}} = \frac{\exp\del{-S\del{\vec x^{(k)}}}}{\int
            \funcmeasure
        \, \exp\del{-S(\vec x)}}
    \]

    detailiertes Gleichgewicht:
    \[
        \frac{W\del{\vec x^{(i)}, \vec x^{(j)}}}{W\del{\vec x^{(j)}, \vec x^{(i)}}}
        = \frac{p\del{\vec x^{(j)}}}{p\del{\vec x^{(i)}}}
    \]
\end{frame}

\begin{frame}
    \frametitle{Metropolisalgorithmus}
    
    \begin{algorithmic}
        \For{$j \gets 1, \ldots, \timesites$}
            \State $x_j' \gets$ \Call{Zufallszahl}{$-\margin$, $\margin$}
            \State $\Deltaup S \gets S(x_0, \ldots, x_{j-1}, x_j',
            x_{j+1}, \ldots, x_\timesites) - S(\{x_i\})$

            \If{$\Deltaup S \leq 0$}
                \State $x_j \gets x_j'$
            \Else
                \State $r \gets$ \Call{Zufallszahl}{0, 1}
                \If{$r < \exp(-\Deltaup S)$}
                    \State $x_j \gets x_j'$
                \EndIf
            \EndIf
        \EndFor
    \end{algorithmic}
\end{frame}

\begin{frame}
    \frametitle{Modifikationen}

    \begin{itemize}
        \item
            Ziehen aus Normalverteilung um $x_j$

        \item
            wiederholtes Ziehen

        \item
            Auslassen von Trajektorien

        \item
            $\Deltaup S$ effizient berechnen
    \end{itemize}
\end{frame}

\section{Energiewerte}

\frame\sectionpage

\subsection{Virialsatz, Korrelationen und GEVP}

\frame\subsectionpage

\begin{frame}
    \frametitle{Virialsatz}

    \begin{itemize}
    
        \item
    Allgemein: $\bracket{\vec F \vec r} + 2 \bracket{T} = 0$

\item
    Mit $\vec F = - \vnabla V$ folgt $2 T = m \bracket{v^2} = \bracket{x V'(x)}$

\item
    Hier: $V(x) = \mu^2 x^2 /2$

\item
    $E_0$ aus verfügbaren Werten:
    \[
        \bracket E = \bracket T + \bracket V
        = \frac12 \mu^2 \bracket{x^2} + \frac12 \mu^2 \bracket{x^2}
        = \mu^2 \bracket{x^2}
    \]
    \end{itemize}
\end{frame}

\begin{frame}
    \frametitle{Korrelationen}

    $E_1$ durch Korrelationen und $E_0$:
    \[
        E_1 = \lim_{\tau \to \infty} \frac{-1}{\Deltaup \tau} \ln
        \del{\frac{\Bracket{x(0) \, x(\tau + \Deltaup \tau)}}{\Bracket{x(0) \,
        x(\tau)}}} + E_0
    \]

    Äquivalent zu
    \[
        f(\tau) := c_1 \exp\del{-\Deltaup E_1 \tau}.
    \]
\end{frame}


\begin{frame}
    \frametitle{Methode von \cite{Blossier/Eigenvalue}}

    
    Korrelationsfunktionen:
    \[
        C_{ij}(\tau) := \Bracket{O_i(\tau) \, O_j^*(0)}
        = \Bracket{x(\tau)^i \, x(0)^j}
    \]

    Bedingungen:
    \begin{itemize}
        \item
            $\braket{0 | O_i | 0}$ liefert keinen Beitrag
        \item
            $E_n < E_{n+1}$
    \end{itemize}
\end{frame}

\begin{frame}
    \frametitle{Generalisiertes Eigenwertproblem}

    GEVP mit $n = 1, \ldots, N$ und $\tau > \tau_0$:
    \[
        C(\tau) \, \vec v_n(\tau, \tau_0) = \lambda_n(\tau, \tau_0) \, 
        C(\tau_0) \, \vec v_n(\tau, \tau_0)
    \]

    Effektive Energiewerte:
    \[
        \Deltaup E_n^\text{eff}(\tau, \tau_0) = -
        \frac 1\timestep \sbr{ \ln\del{\lambda_n(\tau+\timestep, \tau_0)} -
        \ln\del{\lambda_n(\tau, \tau_0)} }
    \]

    Energiewerte:
    \[
        \lim_{\tau \to \infty} \Deltaup E_n^\text{eff}(\tau, \tau_0) = E_n - E_0
    \]

\end{frame}



\begin{frame}
    \frametitle{Choleskyzerlegung}

    \begin{itemize}
        \item
            $A \vec x = \lambda B \vec x$

        \item
            Choleskyzerlegung $B = L L^\dagger$ 

        \item
            Umformungen
            \[
                L^{-1} A [L^\dagger]^{-1} \; L^\dagger \vec x
                =
                \lambda [L^\dagger \vec x].
            \]
            \[
                \tilde{A} := L^{-1} A [L^\dagger]^{-1}
                \eqnsep
                \tilde{\vec x} := L^\dagger \vec x,
            \]

        \item
            normales Eigenwertproblem (gleiche Eigenwerte)
            \[
                \tilde{A} \tilde{\vec x} = \lambda \tilde{\vec x}.
            \]
    \end{itemize}
\end{frame}

\subsection{Oszillator mit $\delta$-Störung}

\frame\subsectionpage

\begin{frame}
    \frametitle{Definition des System}

    Hamiltonoperator:
    \[
        H_\text{rel} = \underbrace{- \frac12 \nabla_x^2 + \frac 12
        x^2}_{H_\text{Osz}} - \frac2{a_0} \delta(x).
    \]

    Schödingergleichung:
    \[
        \sbr{H_\text{Osz} - \frac2{a_0} \delta(x)} \Psi(x) = E \Psi(x)
    \]

    Analytische Lösung durch \cite{Busch/Two_Cold}
\end{frame}

\begin{frame}
    \begin{itemize}
        \item
            Entwicklung in Eigenfunktionen $\phi_n$ des harmonischen Oszillators
        \item
            Projektion auf $\phi_m^*$
        \item
            Ausnutzen der Orthogonalitätsrelation und $\delta$-Distribution
        \item
            Ansatz für Entwicklungskoeffizienten $c_n \propto \phi_m(0)/[E_m - E]$
        \item
            Eigenfunktionen bei $x = 0$ einsetzen
        \item
            Summe ausführen
        \item
            nach $1/a_0$ auflösen ergibt:
            \[
                - \frac{[2E-1]\Gamma\del{\frac 14[3-2E]}}{4 \Gamma\del{\frac
                14[5-2E]}} = \frac1{a_0}.
            \]
    \end{itemize}
\end{frame}

\begin{frame}
    \frametitle{Energiewerte}
    
    \tikzsetnextfilename{E_a0}
    \begin{tikzpicture}
        \begin{axis}[
                width=\linewidth,
                height=0.6\linewidth,
                xmin=-3,
                xmax=3,
                ymax=5.8,
                xlabel=$1/a_0$,
                ylabel=$E$,
                xtick={-1,0,1},
                ytick={-2.5,-1.5,...,9.5},
                minor y tick num=1,
                xmajorgrids=true,
            ]
            \addplot[black] table {../Text/_build/gamma_data.csv};
        \end{axis}
    \end{tikzpicture}
\end{frame}

\section{Ergebnisse}

\frame\sectionpage

\subsection{Harmonischer Oszillator}

\frame\subsectionpage

\begin{frame}
    \frametitle{Trajektorie nach Relaxation}

    \tikzsetnextfilename{relaxiert}
    \begin{tikzpicture}
        \begin{axis}[
                width=\linewidth,
                height=0.6\linewidth,
                xlabel={$j = t/a$},
                ylabel={$x_j = x(t/a)$},
            ]
            \addplot[black] table {../Text/CSV/11EA8F-trajectory-04-more_iterations.csv};
        \end{axis}
    \end{tikzpicture}
\end{frame}

\begin{frame}
    \frametitle{Aufenthaltswahrscheinlichkeit ($a = \num{1.0}$)}

    \tikzsetnextfilename{histogram_10}
    \begin{tikzpicture}
        \begin{axis}[
                width=\linewidth,
                height=0.55\linewidth,
                xlabel={$x$},
                ylabel={$|\psi(x)|^2$},
                grid=major,
                xmin=-3,
                xmax=3,
                legend entries={
                    {Simulation},
                    {Theorie},
                    Kontinuum,
                },
            ]
            \addplot[
                black,
                mark=+,
                myerrorbar,
            ] table[y error index=2] {../Text/CSV/DC5256-histogram-position-resultset.csv};
            \addplot[black] table {../Text/_build/histogram_theory_1.csv};
            \addplot[black, dashed] table {../Text/_build/histogram_theory_continous.csv};
        \end{axis}
    \end{tikzpicture}
\end{frame}

\begin{frame}
    \frametitle{Ungerade Eigenwerte}
    \tikzsetnextfilename{ED7AA2-eigenwerte-ungerade}
    \begin{tikzpicture}
        \begin{semilogyaxis}[
                width=\linewidth,
                height=0.6\linewidth,
                xlabel={$\tau$},
                ylabel={$\lambda_n(\tau)$},
                grid=major,
                legend entries={
                    {$n = 1$},
                    {$n = 3$},
                    {$n = 5$},
                },
            ]
            \addplot[
                Set1_1,
                mark=+,
                myerrorbar,
            ] table[y error index=2] {../Text/CSV/lambdas/ED7AA2-eigenvalue-01.csv};

            \addplot[
                Set1_2,
                mark=+,
                myerrorbar,
            ] table[y error index=2] {../Text/CSV/lambdas/ED7AA2-eigenvalue-03.csv};
            \addplot[
                Set1_3,
                mark=+,
                myerrorbar,
            ] table[y error index=2] {../Text/CSV/lambdas/ED7AA2-eigenvalue-05.csv};
            \addplot[Set1_1] table {../Text/CSV/lambdas/ED7AA2-eigenvalue-01-fit.csv};
            \addplot[Set1_2] table {../Text/CSV/lambdas/ED7AA2-eigenvalue-03-fit.csv};
            \addplot[Set1_3] table {../Text/CSV/lambdas/ED7AA2-eigenvalue-05-fit.csv};
        \end{semilogyaxis}
    \end{tikzpicture}
\end{frame}

\begin{frame}
    \frametitle{Errechnete Energiewerte aus Eigenwerten}
    \begin{tabular}{SS}
        {$n$} & {$E_n$} \\
        \midrule
         1 & 1.50331 +- 0.00075 \\
         2 &   2.429 +- 0.020   \\
         3 &   3.433 +- 0.027   \\
         4 &   4.320 +- 0.064   \\
         5 &   4.990 +- 0.095   \\
         6 &    6.29 +- 0.14    \\
         7 &   6.869 +- 0.100   \\
         8 &    18.5 +- 8.1     \\
         9 &    12.4 +- 1.5     \\
        10 &    1.28 +- 0.11    \\
        11 &   1.186 +- 0.060  
    \end{tabular}
\end{frame}

\subsection{Oszillator mit Störung}

\frame\subsectionpage

\begin{frame}
    \frametitle{Gestörtes Potential}
    \begin{columns}[T]
        \begin{column}{.5\textwidth}
            \tikzsetnextfilename{gausspotential-negativ}
            \begin{tikzpicture}
                \begin{axis}[
                        width=\linewidth,
                        height=\linewidth,
                        xlabel={$x$},
                        ylabel={$V(x)$},
                        grid=major,
                    ]
                    \addplot[black] table {../Text/CSV/gauss-potential/pos/18B9E2-potential.csv}; % 1.00
                    \addplot[black] table {../Text/CSV/gauss-potential/pos/DF05AB-potential.csv}; % 0.800
                    \addplot[black] table {../Text/CSV/gauss-potential/pos/00C58F-potential.csv}; % 0.500
                    \addplot[black] table {../Text/CSV/gauss-potential/pos/53D7EB-potential.csv}; % 0.300
                    \addplot[black] table {../Text/CSV/gauss-potential/pos/6FC501-potential.csv}; % 0.100
                \end{axis}
            \end{tikzpicture}
        \end{column}
        \begin{column}{.5\textwidth}
            \tikzsetnextfilename{gausspotential-positiv}
            \begin{tikzpicture}
                \begin{axis}[
                        width=\linewidth,
                        height=\linewidth,
                        xlabel={$x$},
                        ylabel={$V(x)$},
                        grid=major,
                        yticklabel pos=right,
                    ]
                    \addplot[black] table {../Text/CSV/gauss-potential/neg/029833-potential.csv};
                    \addplot[black] table {../Text/CSV/gauss-potential/neg/156D90-potential.csv};
                    \addplot[black] table {../Text/CSV/gauss-potential/neg/4DD6F0-potential.csv};
                \end{axis}
            \end{tikzpicture}
        \end{column}
    \end{columns}
\end{frame}

\begin{frame}
    \frametitle{$E_0$ in Abhängigkeit von $\timestep$ und $\gausswidth$: $1/a_0 = \num{-1}$}
    \tikzsetnextfilename{mesh-e0}
    \begin{tikzpicture}
        \begin{axis}[
                width=0.9\linewidth,
                height=0.6\linewidth,
                xlabel={$\gausswidth$},
                ylabel={$\timestep$},
                colormap/jet,
                zlabel={$E_0$},
                colorbar,
                grid=major,
                view={-15}{15},
            ]
            \addplot3[
                mesh,
                scatter,
                mesh/ordering=y varies,
            ]
            table {../Text/CSV/raw_data.csv};
        \end{axis}
    \end{tikzpicture}
\end{frame}

\begin{frame}
    \frametitle{$E_0$ in Abhängigkeit von $\timestep$: $1/a_0 = \num{-1}$}
    \tikzsetnextfilename{minus1-fein}
    \begin{tikzpicture}
        \begin{axis}[
                width=\linewidth,
                height=0.6\linewidth,
                xlabel={$a$},
                ylabel={$E_0$},
                grid=major,
                legend entries={
                    {$\gausswidth = \num{0.30}$},
                    {$\gausswidth = \num{0.20}$},
                    {$\gausswidth = \num{0.10}$},
                    {$\gausswidth = \num{0.05}$},
                    {$\gausswidth = \num{0.01}$},
                },
                legend pos=south east,
                myfit/.style={
                    solid,
                },
            ]

            \addplot[
                Set1_1,
                myerrorbar,
                mark=+,
            ] table[y error index=2] {../Text/CSV/-1/a-E0-0.3.csv};
            \addplot[
                Set1_2,
                myerrorbar,
                mark=+,
            ] table[y error index=2] {../Text/CSV/-1/a-E0-0.2.csv};
            \addplot[
                Set1_3,
                myerrorbar,
                mark=+,
            ] table[y error index=2] {../Text/CSV/-1/a-E0-0.1.csv};
            \addplot[
                Set1_4,
                myerrorbar,
                mark=+,
            ] table[y error index=2] {../Text/CSV/-1/a-E0-0.05.csv};
            \addplot[
                Set1_5,
                myerrorbar,
                mark=+,
            ] table[y error index=2] {../Text/CSV/-1/a-E0-0.01.csv};

            \addplot[myfit, Set1_5] table {../Text/CSV/-1/a-E0-fit-0.01.csv};
            \addplot[myfit, Set1_4] table {../Text/CSV/-1/a-E0-fit-0.05.csv};
            \addplot[myfit, Set1_3] table {../Text/CSV/-1/a-E0-fit-0.1.csv};
            \addplot[myfit, Set1_2] table {../Text/CSV/-1/a-E0-fit-0.2.csv};
            \addplot[myfit, Set1_1] table {../Text/CSV/-1/a-E0-fit-0.3.csv};

        \end{axis}
    \end{tikzpicture}
\end{frame}

\begin{frame}
    \frametitle{$\lim_{\timestep \to 0} E_0$ in Abhängigkeit von $\gausswidth$}
    \tikzsetnextfilename{minus1-meta}
    \begin{tikzpicture}
        \begin{axis}[
                width=\linewidth,
                height=0.6\linewidth,
                xlabel={$\gausswidth$},
                ylabel={$E_0$},
                grid=major,
                legend pos=south east,
                legend entries={
                    {Daten},
                    {Extrapolation},
                    {Theorie},
                },
            ]

            \addplot[
                black,
                mark=+,
                myerrorbar,
            ] table[y error index=2] {../Text/CSV/-1/a-E0-meta-daten.csv};

            \addplot[
                mark=+,
                myerrorbar,
                Set1_1,
            ] table[y error index=2] {table1-1.csv};

            \addplot[
                mark=10-pointed star,
                only marks,
                Set1_2,
            ] table {table1-2.csv};

        \end{axis}
    \end{tikzpicture}
\end{frame}

\begin{frame}
    \frametitle{$E_0$ in Abhängigkeit von $\timestep$: $1/a_0 = \num{+1}$}
    \tikzsetnextfilename{plus1-E0-a}
    \begin{tikzpicture}
        \begin{axis}[
                width=\linewidth,
                height=0.6\linewidth,
                xlabel={$a$},
                ylabel={$E_0$},
                grid=major,
                legend entries={
                    {$\gausswidth = \num{0.30}$},
                    {$\gausswidth = \num{0.20}$},
                    {$\gausswidth = \num{0.10}$},
                    {$\gausswidth = \num{0.05}$},
                    {$\gausswidth = \num{0.01}$},
                },
                legend pos=south east,
                myfit/.style={
                    solid,
                },
            ]

            \addplot[
                Set1_1,
                myerrorbar,
                mark=+,
            ] table[y error index=2] {../Text/CSV/+1/a-E0-0.3.csv};
            \addplot[
                Set1_2,
                myerrorbar,
                mark=+,
            ] table[y error index=2] {../Text/CSV/+1/a-E0-0.2.csv};
            \addplot[
                Set1_3,
                myerrorbar,
                mark=+,
            ] table[y error index=2] {../Text/CSV/+1/a-E0-0.1.csv};
            \addplot[
                Set1_4,
                myerrorbar,
                mark=+,
            ] table[y error index=2] {../Text/CSV/+1/a-E0-0.05.csv};
            \addplot[
                Set1_5,
                myerrorbar,
                mark=+,
            ] table[y error index=2] {../Text/CSV/+1/a-E0-0.01.csv};

        \end{axis}
    \end{tikzpicture}
\end{frame}

\begin{frame}
    \frametitle{$E_0$ in Abhängigkeit von $\timestep$: $1/a_0 = \num{+1}$}
    \tikzsetnextfilename{plus1-E0-a-zoom}
    \begin{tikzpicture}
        \begin{axis}[
                width=\linewidth,
                height=0.6\linewidth,
                xlabel={$a$},
                ylabel={$E_0$},
                ymin=-3,
                legend pos=south east,
                grid=major,
                myfit/.style={
                    solid,
                },
                legend entries={
                    {$\gausswidth = \num{0.30}$},
                    {$\gausswidth = \num{0.20}$},
                    {$\gausswidth = \num{0.10}$},
                    {$\gausswidth = \num{0.05}$},
                    {$\gausswidth = \num{0.01}$},
                },
            ]

            \addplot[
                Set1_1,
                myerrorbar,
                mark=+,
            ] table[y error index=2] {../Text/CSV/+1/a-E0-0.3.csv};
            \addplot[
                Set1_2,
                myerrorbar,
                mark=+,
            ] table[y error index=2] {../Text/CSV/+1/a-E0-0.2.csv};
            \addplot[
                Set1_3,
                myerrorbar,
                mark=+,
            ] table[y error index=2] {../Text/CSV/+1/a-E0-0.1.csv};
            \addplot[
                Set1_4,
                myerrorbar,
                mark=+,
            ] table[y error index=2] {../Text/CSV/+1/a-E0-0.05.csv};
            \addplot[
                Set1_5,
                myerrorbar,
                mark=+,
            ] table[y error index=2] {../Text/CSV/+1/a-E0-0.01.csv};

        \end{axis}
    \end{tikzpicture}
\end{frame}

\begin{frame}
    \frametitle{$E_0$ gegen $\gausswidth$: $1/a_0 = \num{+1}$}

    \tikzsetnextfilename{plus1-E0-sigma-0p005}
    \begin{tikzpicture}
        \begin{axis}[
                width=\linewidth,
                height=0.6\linewidth,
                xlabel={$\gausswidth$},
                ylabel={$E_0$},
                grid=major,
                legend pos=south east,
                legend entries={
                    {Daten $\timestep = \num{0.005}$},
                    {Theorie},
                },
            ]

            \addplot[
                black!100!white,
                only marks,
                mark=+,
                error bars/.cd,
                y dir=both,
                y explicit
            ] table[y error index=2] {../Text/CSV/+1/sigma-E0-0.005.csv};

            \addplot[
                mark=10-pointed star,
                only marks,
                Set1_2,
            ] table {table2.csv};

        \end{axis}
    \end{tikzpicture}
\end{frame}

\section*{Zusammenfassung}

\begin{frame}
    \frametitle{Zusammenfassung}

    Methoden:
    \begin{itemize}
        \item
            harmonischer Oszillator und Oszillator mit $\delta$-Störung
        \item
            Pfadintegrale
        \item
            \textsc{Metropolis}algorithmus
        \item
            Korrelationen und generalisiertes Eigenwertproblem
    \end{itemize}

    Ergebnisse:
    \begin{itemize}
        \item
            Aufenthaltswahrscheinlichkeit
        \item
            Energiewerte (harmonisch und mit Störung)
    \end{itemize}
\end{frame}

\begin{frame}
    \frametitle{Fragen}

    Haben Sie Fragen?
\end{frame}

\begin{frame}
    \frametitle{Literatur}

    \printbibliography

    \begin{small}
    Zur Übersicht sind nicht alle Quellen angegeben, bitte schauen Sie in der
    Bachelorarbeit selbst ins Quellenverzeichnis.
    \end{small}
\end{frame}

\end{document}

% vim: spell spelllang=de
