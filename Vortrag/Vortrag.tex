\documentclass[ngerman, fleqn]{beamer}

\usetheme{Dresden}

\usepackage{header}

\usepackage{tikz}
\usepackage{pgfplots}
\pgfplotsset{
    compat=1.9,
    width=0.8\linewidth,
    xticklabel style={/pgf/number format/use comma},
    yticklabel style={/pgf/number format/use comma},
}
\usepgfplotslibrary{external}
\tikzexternalize[mode=list and make]
\tikzsetexternalprefix{_build/Abbildung-}

\newcommand\bootstrapsamples{N_\text P}
\newcommand\gausswidth{\sigma}
\newcommand\initialrandomwidth{\tilde \Delta}
\newcommand\iterations{M}
\newcommand\margin{\Delta}
\newcommand\mass{\mu}
\newcommand\preiterations{M'}
\newcommand\rounds{\bar n}
\newcommand\timesites{N}
\newcommand\timestep{a}

\newcommand\funcmeasure{\mathcal D \vec x}


\hypersetup{
    pdftitle={%
        Harmonischer Oszillator in der Pfadintegralformulierung in Euklidischer
        Raum-Zeit mit gitterfeldtheoretischen Methoden
    }
}


\title{Gitterfeldtheoretische Behandlung des Harmonischen Oszillators in der Pfadintegralformulierung in Euklidischer Raum-Zeit}
\subtitle{Präsentation zur Bachelorarbeit}
\author{Martin Ueding \\ mu@martin-ueding.de}
\date{Juli 2014}

\begin{document}

\begin{frame}
    \titlepage
\end{frame}

\begin{frame}
    \tableofcontents
\end{frame}

\section{Pfadintegrale}

\section{Energiewerte}

\subsection{Generalisiertes Eigenwertproblem}

\subsection{Eindimensionale Wellenfunktion mit $\delta$-Störung}

\begin{frame}
    Leer
\end{frame}


\section{Ergebnisse}

\subsection{Harmonischer Oszillator}

\begin{frame}
    \frametitle{Test Titel}
    Test 

    \begin{itemize}
        \item Test
        \item Item
    \end{itemize}

    \parencite{penrose-road_to_reality}
\end{frame}

\begin{frame}
    \frametitle{Ungerade Eigenwerte}
    \tikzsetnextfilename{ED7AA2-eigenwerte-ungerade}
    \begin{tikzpicture}
        \begin{semilogyaxis}[
                width=\linewidth,
                height=0.55\linewidth,
                xlabel={$\tau$},
                ylabel={$\lambda_n(\tau)$},
                grid=major,
                legend entries={
                    {$n = 1$},
                    {$n = 3$},
                    {$n = 5$},
                },
            ]
            \addplot[black!100!white] table {../Text/CSV/lambdas/ED7AA2-eigenvalue-01-fit.csv};
            \addplot[black!60!white] table {../Text/CSV/lambdas/ED7AA2-eigenvalue-03-fit.csv};
            \addplot[black!30!white] table {../Text/CSV/lambdas/ED7AA2-eigenvalue-05-fit.csv};
            \addplot[
                black!100!white,
                only marks,
                mark=|,
                error bars/.cd,
                y dir=both,
                y explicit
            ] table[y error index=2] {../Text/CSV/lambdas/ED7AA2-eigenvalue-01.csv};

            \addplot[
                black!60!white,
                only marks,
                mark=|,
                error bars/.cd,
                y dir=both,
                y explicit
            ] table[y error index=2] {../Text/CSV/lambdas/ED7AA2-eigenvalue-03.csv};
            \addplot[
                black!30!white,
                only marks,
                mark=|,
                error bars/.cd,
                y dir=both,
                y explicit
            ] table[y error index=2] {../Text/CSV/lambdas/ED7AA2-eigenvalue-05.csv};
        \end{semilogyaxis}
    \end{tikzpicture}
\end{frame}

\subsection{Anharmonischer Oszillator}

\begin{frame}
    \printbibliography
\end{frame}

\end{document}

% vim: spell spelllang=de
